\documentclass[a4paper]{article}

\usepackage[utf8]{inputenc}
\usepackage{multicol}

\title{Working Title}

\author{Chris Morgan}

\date{\today}

\begin{document}
\maketitle

\begin{abstract}
With an increasing percentage of the population living a low intensity and sedentary lifestyle, there is a need to motivate these people to engage in physical exercise. Not only is physical wellbeing affected by a lack of exercise, mental capacity also diminishes faster over time. Exergames are not a new solution to this issue, but most in existence try to emulate the physical activity rather than use it solely as a game mechanic.

We propose to develop an exergame system that is not grounded in reality, while still using a familiar exercise in a relatable context. The game will include an integrated scaling system that balances solo gameplay whilst playing. There will also be multiplayer options to engage with people motivated by competition with others rather than oneself.

To evaluate the exergame we developed, user testing was carried out and compared with other means of physical exercise. This was done through both exergames and real world activity. The exergame proved to be effective at providing them with an enjoyable videogame that incorporates physical activity. It also increased their desire to exercise outside of the study. Additional observations show that exergames less attached to reality, while still being recognisable as exercise, were more enjoyable and motivating than games with closer ties to reality.
\end{abstract}

\section{Introduction}
\begin{multicols}{2}

Regular exercise of moderate intensity is a proven means to reduce incidence of obesity as well as provide benefits for mental capacity and fluidity. Even with these benefits, only half the population are partaking in regular exercise \cite{ministry2013new}. This could be one of the reasons for just under a third of the population being obese and a further third overweight \cite{ministry2013new}.

With obesity and related ailments alone estimated cost \$8 billion over the next decade \cite{lal2012health}, a lack of exercise is not only causing problems with our health but is putting major strain on the economy. A consequence of exercise not being motivating \cite{bartholomew2005college} is that individuals are not exercising enough or at a level of sufficient intensity.

Exergaming is a genre of videogames that integrate exercise into the core gameplay. The genre is an attempt to distract users from the exercise as a means to increase motivation. Research has been done into what motivates people to exercise, whether goal based games are effective \cite{jin2010does}, performance differences in exergaming vs pure exercise \cite{kraft2011heart,maddison2009feasibility}, among other areas. Current research is lacking a solution to problem with motivation. In particular, studies are lacking in the area of competitive multiplayer.

In this paper we will be filling in these gaps in the research, focusing on competitive, multiplayer exergames and the behaviour of individuals whilst playing them.

A few questions arise from the idea of multiplayer exergames:

\begin{enumerate}
	\item Does competitive multiplayer increase motivation and/or performance in exergames?
	\item Are people more motivated by/perform better with live competition or a highscore orientated goal.
\end{enumerate}

Evaluating how individuals react to competition in exergames would provide insight into how to develop exergames that are more engaging for users. 

Testing the difference in performance and motivation of players between two competitive systems is a complicated problem in itself to solve. The two systems will involve competing against another player or competing against the `ghost' of the current high score. In reality, in both systems the test subject will be competing against the `ghost' of a previous run by someone else. This placebo is to remove as many variables from the experiment as possible, boiling it down to how people respond to different competitive situations.

\end{multicols}

\bibliographystyle{ieeetr}
\bibliography{References}

\end{document}